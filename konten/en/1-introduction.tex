% Ubah judul dan label berikut sesuai dengan yang diinginkan.
\section{Introduction}
\label{sec:introduction}

The increasing use of trucks for goods transportation in Indonesia presents significant challenges, particularly regarding overdimension vehicles. According to the Central Bureau of Statistics \cite{bps2023}, trucks constitute the third largest vehicle category, but their operation often leads to safety concerns. Studies show that ODOL (Overdimension Overloading) vehicles contribute to 32\% of toll road accidents and reduce average vehicle speeds by 12\% \cite{odol2020}.

Current manual inspection methods by UPPKB (Motor Vehicle Weighing Implementation Unit) cover only 5\% of vehicles, with 27.95\% of inspected vehicles found violating regulations - 69\% for excess cargo and 31\% for documentation \cite{hubdat2024}. This limited coverage highlights the need for automated detection systems.

This research aims to develop an automated overdimension vehicle detection system using deep learning, specifically Convolutional Neural Network (CNN), implemented on edge devices. The system focuses on:
\begin{itemize}
  \item Real-time detection on edge devices
  \item Scalable integration with existing systems
  \item Automatic violation notifications
\end{itemize}

The study is limited to specific deep learning models and controlled testing environments but aims to provide a foundation for broader implementation across various locations.

The benefits obtained from this research include contributing to the development of overdimension vehicle detection technology using deep learning models, providing effective and efficient solutions in detecting overdimension vehicles in real-time, and providing recommendations for further research in the development of overdimension vehicle detection systems.