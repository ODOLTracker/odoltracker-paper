% Mengubah keterangan `Abstract` ke bahasa indonesia.
% Hapus bagian ini untuk mengembalikan ke format awal.
% \renewcommand\abstractname{Abstrak}

\begin{abstract}

  % Ubah paragraf berikut sesuai dengan abstrak dari penelitian.
  The use of trucks as a mode of transportation for goods in Indonesia continues to increase, but violations of ODOL (Overdimension Overloading) on trucks have become a major cause of traffic accidents. This research aims to develop an overdimension vehicle detection system using deep learning technology, specifically Convolutional Neural Network (CNN) with SSD-MobileNetV2 architecture, that can operate in real-time on edge devices. The model was trained with annotated vehicle datasets, achieving a mAP value of 0.805 and detection accuracy of 80.72\%. Testing on two types of edge devices showed that NVIDIA Jetson Nano provided the best performance with an inference speed of 46.86 FPS, significantly higher than Beelink Gemini T34 (3.63 FPS). The system was implemented at Dupak 2 Toll Gate, Surabaya with a 100\% data transfer success rate despite bandwidth limitations. Compared to similar research, this system has advantages in balancing accuracy and speed, as well as bandwidth efficiency(4 KB/s). The system is also integrated with a cloud backend and equipped with automatic notification features to authorities when violations are detected. With improved accuracy and efficiency, this system provides an effective solution for detecting ODOL vehicles in various locations with connectivity limitations.

\end{abstract}

% Mengubah keterangan `Index terms` ke bahasa indonesia.
% Hapus bagian ini untuk mengembalikan ke format awal.
% \renewcommand\IEEEkeywordsname{Kata kunci}

\begin{IEEEkeywords}

  % Ubah kata-kata berikut sesuai dengan kata kunci dari penelitian.
  Overdimension, ODOL, deep learning, SSD-MobileNetV2, edge device, real-time detection, cloud integration, Jetson Nano

\end{IEEEkeywords}
